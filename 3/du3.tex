\documentclass[12pt, a4paper]{article}
\usepackage[margin=1in]{geometry}
\usepackage[utf8x]{inputenc}
\usepackage{indentfirst} %indentace prvního odstavce
\usepackage{mathtools}
\usepackage{amsfonts}
\usepackage{amsmath}
\usepackage{amssymb}
\usepackage{graphicx}
\usepackage{enumitem}
\usepackage{subfig}
\usepackage{float}
\usepackage[czech]{babel}
\usepackage{mathdots}
\usepackage{slashbox}
\newcommand{\qed}{\hfill\square}
\begin{document}
\begin{center}
\large NMMB331 - HW3

\normalsize Jan Oupický
\end{center}
\vspace{1\baselineskip}

\section{}
Let $x \in \mathbb{F}^n_2$. Then
\begin{gather*}
\frac{1}{2^n}\sum\limits_{u \in \mathbb{F}^n_2} F_g(u)(-1)^{\langle u, x \rangle} = \frac{1}{2^n}\sum\limits_{u \in \mathbb{F}^n_2} \sum\limits_{y \in \mathbb{F}^n_2}g(y)(-1)^{\langle u, y \rangle} (-1)^{\langle u, x \rangle} = \\
\frac{1}{2^n}\sum\limits_{y \in \mathbb{F}^n_2} g(y)\sum\limits_{u \in \mathbb{F}^n_2}(-1)^{\langle u, x+y \rangle}\\
\text{There is exactly one $y$ s.t. $x+y=0 \iff x=y$. Using annihilator lemma we get:}\\
\frac{1}{2^n}g(x)2^n + 0 = g(x)
\end{gather*}
$\qed$

\section{}
First assume $u=0$.
\begin{gather*}
F_f(0)=\sum\limits_{x \in \mathbb{F}^n_2}f(x)(-1)^{\langle 0,x \rangle} = \sum\limits_{x \in \mathbb{F}^n_2}f(x) = \sum\limits_{x \in \mathbb{F}^n_2, f(x)=1}1\\
2^{n-1}-\frac{1}{2}W_f(0) = \frac{1}{2}\left(\sum\limits_{x \in \mathbb{F}^n_2}(1-(-1)^{f(x)})\right) = \frac{1}{2}\left(\sum\limits_{x \in \mathbb{F}^n_2, f(x)=1}(1-(-1))\right)=\\
\frac{1}{2}\sum\limits_{x \in \mathbb{F}^n_2, f(x)=1}2 = F_f(0)
\end{gather*}
Now $u \neq 0$:
\begin{gather*}
F_f(u)=\sum\limits_{x \in \mathbb{F}^n_2}f(x)(-1)^{\langle u,x \rangle} = \sum\limits_{x \in \mathbb{F}^n_2,f(x)=0}0(-1)^{\langle u,x \rangle} + \sum\limits_{x \in \mathbb{F}^n_2,f(x)=1}1(-1)^{\langle u,x \rangle} =\\
\sum\limits_{x \in \mathbb{F}^n_2,f(x)=1}(-1)^{\langle u,x \rangle}\\
-\frac{1}{2}W_f(u) =\frac{1}{2}\left( \sum\limits_{x \in \mathbb{F}^n_2}(-1)^{\langle u,x \rangle} - W_f(u) \right) = \frac{1}{2}\left( \sum\limits_{x \in \mathbb{F}^n_2}(-1)^{\langle u,x \rangle} - (-1)^{f(x)+\langle u,x \rangle} \right) = \\
\frac{1}{2}\left( \sum\limits_{x \in \mathbb{F}^n_2}(-1)^{\langle u,x \rangle}(1 - (-1)^{f(x)} \right) = \frac{1}{2}\left( \sum\limits_{x \in \mathbb{F}^n_2, f(x)=0}(-1)^{\langle u,x \rangle}(1 - 1) \right) + \\
\frac{1}{2}\left( \sum\limits_{x \in \mathbb{F}^n_2, f(x)=1}(-1)^{\langle u,x \rangle}(1 - (-1)) \right) = 0 + F_f(u) = F_f(u)
\end{gather*}
$\qed$

\section{}
\begin{gather*}
F_{f \oplus g}(u) = \sum\limits_{x \in \mathbb{F}^n_2}(f \oplus g)(x)(-1)^{\langle u,x \rangle} = \sum\limits_{x \in \mathbb{F}^n_2}(f(x) \oplus g(x))(-1)^{\langle u,x \rangle} = \\
\sum\limits_{x \in \mathbb{F}^n_2}(f(x) + g(x) - 2f(x)g(x))(-1)^{\langle u,x \rangle} = \\
\sum\limits_{x \in \mathbb{F}^n_2}f(x)(-1)^{\langle u,x \rangle} + \sum\limits_{x \in \mathbb{F}^n_2}g(x)(-1)^{\langle u,x \rangle} + \sum\limits_{x \in \mathbb{F}^n_2}2f(x)g(x)(-1)^{\langle u,x \rangle} = \\
F_f(u) + F_g(u) + \sum\limits_{x \in \mathbb{F}^n_2}2(fg)(x)(-1)^{\langle u,x \rangle} = F_f(u) + F_g(u) + F_{2fg}(u)
\end{gather*}
$\qed$

\section{}
\begin{gather*}
\sum\limits_{u \in \mathbb{F}^n_2} (F_f(u))^2 = \sum\limits_{u \in \mathbb{F}^n_2} \left( \sum\limits_{x \in \mathbb{F}^n_2} f(x)(-1)^{\langle u,x \rangle} \right)^2 =\\
\sum\limits_{u \in \mathbb{F}^n_2} \sum\limits_{x \in \mathbb{F}^n_2} \sum\limits_{y \in \mathbb{F}^n_2} f(x)f(y)(-1)^{\langle u,x \rangle} (-1)^{\langle u,y \rangle}=\\
\sum\limits_{x \in \mathbb{F}^n_2} \sum\limits_{y \in \mathbb{F}^n_2} f(x)f(y) \sum\limits_{u \in \mathbb{F}^n_2} (-1)^{\langle u,x+y \rangle}\\
\text{Using annihilator lemma there is again only one $y$ s.t. $x+y=0 \iff x=y$:}\\
\sum\limits_{x \in \mathbb{F}^n_2} \sum\limits_{y \in \mathbb{F}^n_2, y=x} f(x)f(y)2^n = \sum\limits_{x \in \mathbb{F}^n_2} f(x)^2 2^n = 2^n\sum\limits_{x \in \mathbb{F}^n_2} f(x) = 2^nw_t(f)\\
\end{gather*}

\section{}
\begin{gather*}
\frac{1}{2^n}\sum\limits_{v \in \mathbb{F}^n_2} (F_f(v))^2(-1)^{\langle u,v \rangle} = \frac{1}{2^n}\sum\limits_{v \in \mathbb{F}^n_2} \sum\limits_{x \in \mathbb{F}^n_2} \sum\limits_{y \in \mathbb{F}^n_2} f(x)f(y)(-1)^{\langle v,x \rangle}(-1)^{\langle v,y \rangle}(-1)^{\langle u,v \rangle} =\\
\frac{1}{2^n}\sum\limits_{x \in \mathbb{F}^n_2} \sum\limits_{y \in \mathbb{F}^n_2} f(x)f(y) \sum\limits_{v \in \mathbb{F}^n_2} (-1)^{\langle v,x+y \rangle}(-1)^{\langle u,v \rangle} =
\frac{1}{2^n}\sum\limits_{x \in \mathbb{F}^n_2} \sum\limits_{y \in \mathbb{F}^n_2} f(x)f(y) \sum\limits_{v \in \mathbb{F}^n_2} (-1)^{\langle v,x+y+u \rangle} \\
\text{Annihilator lemma $x+y+u=0 \iff y=x+u$:}\\
\frac{1}{2^n}\sum\limits_{x \in \mathbb{F}^n_2} \sum\limits_{y \in \mathbb{F}^n_2, y=x+u} f(x)f(y) 2^n = \sum\limits_{x \in \mathbb{F}^n_2} \sum\limits_{y \in \mathbb{F}^n_2, y=x+u} f(x)f(y) = 
\sum\limits_{x \in \mathbb{F}^n_2} f(x)f(x+u) = A_f(u)
\end{gather*}

\section{}
First assume we have $u \in \mathbb{F}^n_2$ and $W_f(v)=0$ for all $v \in \mathbb{F}^n_2: \langle u,v \rangle = 0$. Let's use the inverse fourier transform on the sum $f(x)+f(x+u)$ for $x \in \in \mathbb{F}^n_2$:
\begin{gather*}
f(x)+f(x+u) = \frac{1}{2^n}\sum\limits_{v \in \mathbb{F}^n_2} F_f(v)(-1)^{\langle v,x \rangle}+ \frac{1}{2^n}\sum\limits_{v \in \mathbb{F}^n_2} F_f(v)(-1)^{\langle v,x+u \rangle} = \\
\frac{1}{2^n}\sum\limits_{v \in \mathbb{F}^n_2} F_f(v)((-1)^{\langle v,x \rangle}+(-1)^{\langle v,x \rangle}(-1)^{\langle v,u \rangle}) = \frac{1}{2^n}\sum\limits_{v \in \mathbb{F}^n_2} F_f(v)(-1)^{\langle v,x \rangle}(1+(-1)^{\langle v,u \rangle})\\
\text{For $v$ s.t. $\langle u,v \rangle = 1$ the summands are 0, so we can only take into account $v: \langle u,v \rangle = 0$}\\
f(x)+f(x+u) = \frac{1}{2^n}\sum\limits_{v \in \mathbb{F}^n_2, \langle u,v \rangle = 0} 2F_f(v)(-1)^{\langle v,x \rangle}\\
\end{gather*}
Since we assume $W_f(v)=0$ for all $v: \langle u,v \rangle = 0$ then for $v\neq 0$ it implies that $F_f(v) = 0$. 
So we only consider $v=0$:
\begin{gather*}
f(x)+f(x+u) = \frac{1}{2^{n-1}}F_f(0) = \frac{1}{2^{n-1}}(2^{n-1}-\frac{1}{2}W_f(0)) = \frac{1}{2^{n-1}}(2^{n-1}-0) = 1
\end{gather*}
This proves the implication.

Now consider there exists $u \in \mathbb{F}^n_2: \forall x \in \in \mathbb{F}^n_2: f(x)+f(x+u)=1$. This means that $f$ is balanced since for every $x$ there must exist $x+u$ s.t. $f(x) \neq f(x+u)$. If $f$ is balanced we have $wt(f)=2^{n-1}$ and $W_f(0)=0$. Also $A_f(u)=0$ because:
\begin{gather*}
A_f(u)=\sum\limits_{x \in \in \mathbb{F}^n_2} f(x)f(x+u) = \sum\limits_{x \in \in \mathbb{F}^n_2} f(x)(1+f(x)) = 0
\end{gather*}

Let's use previous exercices:
\begin{gather*}
0=A_f(u)= \frac{1}{2^n} \sum\limits_{v  \in \mathbb{F}^n_2} (F_f(v))^2 (-1)^{\langle u,v \rangle} \implies \\
0 = \sum\limits_{v  \in \mathbb{F}^n_2} (F_f(v))^2 (-1)^{\langle u,v \rangle} = \sum\limits_{v  \in \mathbb{F}^n_2, \langle v,u \rangle = 0} (F_f(v))^2 - \sum\limits_{v  \in \mathbb{F}^n_2, \langle v,u \rangle = 1} (F_f(v))^2 =\\
\sum\limits_{v  \in \mathbb{F}^n_2, \langle v,u \rangle = 0} (F_f(v))^2 - \left(2^n wt(f) - \sum\limits_{v  \in \mathbb{F}^n_2, \langle v,u \rangle = 0} (F_f(v))^2 \right) =\\
\sum\limits_{v  \in \mathbb{F}^n_2, \langle v,u \rangle = 0} 2(F_f(v))^2 - 2^n2^{n-1} = 0 \iff 0 = \sum\limits_{v  \in \mathbb{F}^n_2, \langle v,u \rangle = 0} (F_f(v))^2 - 2^{2n-2} =\\
F_f(0)^2 + \sum\limits_{v  \in \mathbb{F}^n_2, v \neq 0, \langle v,u \rangle = 0}(F_f(v))^2 - 2^{2n-2} = (2^{n-1}-\frac{1}{2}W_f(0))^2 + \sum\limits_{v  \in \mathbb{F}^n_2, v \neq 0, \langle v,u \rangle = 0}(F_f(v))^2 - 2^{2n-2}\\
\text{$W_f(0) = 0$:}\\
0 = 2^{2n-2} + \sum\limits_{v  \in \mathbb{F}^n_2, v \neq 0, \langle v,u \rangle = 0}(F_f(v))^2 - 2^{2n-2} \iff \sum\limits_{v  \in \mathbb{F}^n_2, v \neq 0, \langle v,u \rangle = 0}(W_f(v))^2 = 0
\end{gather*}
Since all summands are positive the only possibility is that $\forall v \neq 0, \langle u,v \rangle = 0$ it holds that $W_f(v)=0$. The case left is when $v=0$. We have actually already proved it since $f$ is balanced $\implies W_f(0)=0$. 
\end{document}

