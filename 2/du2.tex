\documentclass[12pt, a4paper]{article}
\usepackage[margin=1in]{geometry}
\usepackage[utf8x]{inputenc}
\usepackage{indentfirst} %indentace prvního odstavce
\usepackage{mathtools}
\usepackage{amsfonts}
\usepackage{amsmath}
\usepackage{amssymb}
\usepackage{graphicx}
\usepackage{enumitem}
\usepackage{subfig}
\usepackage{float}
\usepackage[czech]{babel}
\usepackage{mathdots}
\usepackage{slashbox}

\begin{document}
\begin{center}
\large NMMB331 - HW2

\normalsize Jan Oupický
\end{center}
\vspace{1\baselineskip}

\section{}
Let's assume $L: \mathbb{F}_2^n \rightarrow \mathbb{F}_2^m$. Then
\begin{gather*}
\text{Im}(L^*)^\perp = \{y \in \mathbb{F}_2^n | \forall x \in \text{Im}(L^*) : \langle x,y \rangle = 0\} = \{y \in \mathbb{F}_2^n | \forall x \in \mathbb{F}_2^m : \langle L^*(x),y \rangle = 0\}\\
\text{From lecture we know that $\forall x,y:  \langle L^*(x),y \rangle =  \langle x,L(y) \rangle$} \implies\\
\text{Im}(L^*)^\perp = \{y \in \mathbb{F}_2^n | \forall x \in \mathbb{F}_2^m : \langle x,L(y) \rangle = 0\}
\end{gather*}
We know that for a fixed $v \in \mathbb{F}_2^m$ if $v$ satisfies $\forall x \in \mathbb{F}_2^m: \langle x,v \rangle = 0$ then $v = \underline{0}$. Using this fact we can see that $\text{Im}(L^*)^\perp$ is exactly those $y \in \mathbb{F}_2^n$ such that $L(y) = 0$. That is the definition of $\text{Ker}(L)$.

\section{}
Let $G,F$ be EA-equivalent vectorial boolean functions i.e.:
\begin{gather}
G = A_1 \circ F \circ A_2 + A_3 \text{ where for $i=1,2,3$: } A_i(x) = L_i(x) + b
\end{gather}
and $L_i$ is a linear permutation. As in the lecture, choose $u,v$ and we will show that $|\hat{G}(u,v)|$ corresponds to $|\hat{F}(a,b)|$ for exactly one pair $(a,b)$.
Note that we can write:
\[G(x) = L_1 \circ F (L_2(x)+b_2) + b_1 + L_3(x)+b_3\]
Denote $z = L_2(x) + b_2$ then $x = L_2^{-1}(z-b_2) = L_2^{-1}(z) + L_2^{-1}(b_2)$ ($L_2$ is a linear permutation).
\begin{gather*}
\hat{G}(u,v) = \sum\limits_{x \in \mathbb{F}_2^n} \chi(u \cdot G(x) + v\cdot x) = \sum\limits_{x \in \mathbb{F}_2^n} \chi(u \cdot (L_1 \circ F(z) + b_1 + L_3(x)+b_3) + v\cdot x) = \\
\chi(u\cdot(b_1+b_3)) \sum\limits_{z \in \mathbb{F}_2^n} \chi(u \cdot (L_1 \circ F(z) + L_3 \circ L_2^{-1}(z) +L_3 \circ L_2^{-1}(b_2))) + v\cdot (L_2^{-1}(z) +L_2^{-1}(b_2))) =\\
\end{gather*}
Denote $c = \chi(u\cdot(b_1+b_3 + L_3\circ L_2^{-1}(b_2)) + v\cdot L_2^{-1}(b_2))$. By definition $c = \pm 1$. So we can ommit it.
\begin{gather*}
|\hat{G}(u,v)| = \sum\limits_{z \in \mathbb{F}_2^n} \chi(u \cdot (L_1 \circ F(z) + L_3 \circ L_2^{-1}(z)) + v\cdot L_2^{-1}(z)) =\\
\sum\limits_{z \in \mathbb{F}_2^n} \chi(u \cdot L_1 \circ F(z) + u\cdot L_3 \circ L_2^{-1}(z) + v\cdot L_2^{-1}(z)) = \\
\sum\limits_{z \in \mathbb{F}_2^n} \chi(L_1^*(u) \cdot F(z) + (L_3 \circ L_2^{-1})^*(u) \cdot z + (L_2^{-1})^*(v)\cdot z) = \\
\sum\limits_{z \in \mathbb{F}_2^n} \chi(L_1^*(u) \cdot F(z) + ((L_3 \circ L_2^{-1})^*(u) + (L_2^{-1})^*(v))\cdot z) \implies\\
|\hat{G}(u,v)| = |\hat{F}(a,b)| \text{ where }\\
a = L_1^*(u)\\
b = (L_3 \circ L_2^{-1})^*(u) + (L_2^{-1})^*(v)
\end{gather*}
$a,b$ are determined uniquely since $L_i$s (and equivalently $L_i^*$s) are permutations.

\section{}
Let $F: \mathbb{F}_2^n \rightarrow \mathbb{F}_2^n$ vectorial boolean function. Denote $f(x) = \sum\limits_{i = 0}^{2^n-1}a_ix^i \in \mathbb{F}_{2^n}[x]$ it's polynomial form.

Since we know that $F$ is a boolean function iff $\forall x \in \mathbb{F}_2^n: F(x)=(F(x))^2$ it must hold that in that case $f(x) = (f(x))^2 \in \mathbb{F}_{2^n}$. Let's expand this equality using the properties of $\mathbb{F}_{2^n}$. For $x,y \in \mathbb{F}_{2^n}: (x+y)^2 = x^2+y^2$ and $x^{2^n} = x$.
\begin{gather*}
f(x) = \sum\limits_{i = 0}^{2^n-1}a_ix^i = a_0 + a_1x + a_2x^2 + \dots + a_{2^n-1}x^{2^n-1} \\
(f(x))^2 = \left( \sum\limits_{i = 0}^{2^n-1}a_ix^i \right)^2 = \sum\limits_{i = 0}^{2^n-1}a_i^2x^{2i} \implies\\
(f(x))^2 = a_0^2 + a_1^2x^2 + a_2^2x^4 + \dots + a_{\frac{2^n-2}{2}}^2x^{2^n-2} + a_{\frac{2^n-2}{2}+1}^2x^{2^n} + \dots + a_{2^n-1}^2x^{2(2^n-1)}
\end{gather*}
Since $x^{2^n} = x$ the coefficients $a_i$ with $2^n > i > \frac{2^n-2}{2}$ are coefficients at some $x^j$ where $j$ is odd and $j < 2^n$. Therefore rewritten:
\begin{gather*}
(f(x))^2 = a_0^2 + a_{\frac{2^n-2}{2}+1}^2x + a_1^2x^2 + \dots + a_{\frac{2^n-2}{2}}^2x^{2^n-2} + a_{2^n-1}^2x^{2^n-1}
\end{gather*}
And we need that $f(x) = (f(x))^2$ so we compare coefficients at $x^i$ this gives us $a_0^2 = a_0$ and $a_{2^n-1}^2 = a_{2^n-1}$. The only elements of $\mathbb{F}_{2^n}$ for which this holds are $0,1$ therefore it must be that $a_0,a_{2^n-1} \in \mathbb{F}_2 \subseteq \mathbb{F}_{2^n}$. We have also conditions on the rest of the coefficients $a_i$ where $1 \leq i \leq 2^n-2$. One way to write it is that $1 \leq i \leq 2^n-2: a_i^2=a_{2i \text{ mod } 2^n-1}$.

Every polynomial with coefficients that fulfill these conditions must correspond to a vectorial boolean function that $\forall x: F(x)^2=F(x)$ and this means that $F$ can be considered a boolean function since it "outputs" only elements of the subfield $\mathbb{F}_2 \subseteq \mathbb{F}_{2^n}$.

\end{document}

