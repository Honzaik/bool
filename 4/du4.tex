\documentclass[12pt, a4paper]{article}
\usepackage[margin=1in]{geometry}
\usepackage[utf8x]{inputenc}
\usepackage{indentfirst} %indentace prvního odstavce
\usepackage{mathtools}
\usepackage{amsfonts}
\usepackage{amsmath}
\usepackage{amssymb}
\usepackage{graphicx}
\usepackage{enumitem}
\usepackage{subfig}
\usepackage{float}
\usepackage[czech]{babel}
\usepackage{mathdots}
\usepackage{slashbox}
\newcommand{\qed}{\hfill\square}
\begin{document}
\begin{center}
\large NMMB331 - HW4

\normalsize Jan Oupický
\end{center}
\vspace{1\baselineskip}

\section{}
\begin{enumerate}
    \item $\sim_{EA}$ is reflexive since we can take $A=id=B$ (identities) and $C = 0$ (null map).
    \item If $A,B$ are affine permutations and $C$ is an affine map s.t. $G(x)= A \circ F \circ B (x) + C(x) \iff F \sim_{EA} G$ then we can write $A^{-1}\circ G \circ B^{-1} (x) + A^{-1}\circ C \circ B^{-1} (x) = F(x) \iff G \sim_{EA} F$. Since $A,B$ are affine permutations then their inverses are also affine permutations and also $A^{-1}\circ C \circ B^{-1}$ is affine since $C$ is affine and $A^{-1},B^{-1}$ are affine permutations. This proves symmetry.
    \item Assume $F \sim_{EA} G$, $G \sim_{EA} H$ and we want to show $F \sim_{EA} H$. By definition we have $G(x)= A_1 \circ F \circ B_1 (x) + C_1(x)$ and $H(x) = A_2 \circ G \circ B_2 (x) + C_2(x)$. $A_2(x) = M(x)+y$ for some $M$ linear permutation and $y$ vector.
    \begin{gather*}
    H(x) = A_2 \circ G \circ B_2 (x) + C_2(x) = A_2 (A_1 \circ F \circ B_1 (B_2(x)) + C_1(B_2(x))) + C_2(x) =\\
    M(A_1 \circ F \circ B_1 (B_2(x)) + C_1(B_2(x))) + y + C_2(x) =\\
    M(A_1 \circ F \circ B_1 (B_2(x))) + M(C_1(B_2(x))) + y + C_2(x) = \\
    (M \circ A_1) \circ F \circ (B_1 \circ B_2)(x) + (M \circ C_1 \circ B_2(x) + y + C_2(x))
    \end{gather*}
    $M \circ A_1$ is an affine permutation since $M$ is linear permutation and $A_1$ is affine permutation. $B_1 \circ B_2$ is affine permutation since both are affine permutations. $M \circ C_1 \circ B_2$ is affine map since $M, B_2$ are affine permutations and $C_1$ affine map, $C_2(x)+y$ is an affine map. Sum of affine maps is affine map. This proves transitivity.
\end{enumerate}
\end{document}

