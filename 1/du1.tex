\documentclass[12pt, a4paper]{article}
\usepackage[margin=1in]{geometry}
\usepackage[utf8x]{inputenc}
\usepackage{indentfirst} %indentace prvního odstavce
\usepackage{mathtools}
\usepackage{amsfonts}
\usepackage{amsmath}
\usepackage{amssymb}
\usepackage{graphicx}
\usepackage{enumitem}
\usepackage{subfig}
\usepackage{float}
\usepackage[czech]{babel}
\usepackage{mathdots}
\usepackage{slashbox}

\begin{document}
\begin{center}
\large NMMB331 - HW1

\normalsize Jan Oupický
\end{center}
\vspace{1\baselineskip}

\section{}
a

\section{}
a

\section{}
Let's denote $x = \gcd(m,d), y = \gcd(2^m-1, 2^d-1)$. We know that $y | 2^m-1$ and $y | 2^d-1 \implies 2^m \equiv 1 \, (y), 2^n \equiv 1 \, (y) \implies ord_{\mathbb{Z}_y}(2)|m, ord_{\mathbb{Z}_y}(2)|d \implies ord_{\mathbb{Z}_y}(2)|\gcd(m,d)=x$. Therefore $2^x \equiv 1 \, (y) \iff \gcd(2^m-1, 2^d-1) | 2^x-1$.

Let's denote $x = \gcd(m,d), y = \gcd(2^m-1, 2^d-1)$. Assume that $a | 2^m-1, 2^d-1 \iff 2^m \equiv 1 \, (a), 2^n \equiv 1 \, (a) \iff ord_{\mathbb{Z}_a}(2)|m,d \iff ord_{\mathbb{Z}_a}(2)|\gcd(m,d)=x \iff 2^x \equiv 1 \, (a)$. There have been equivalences everywhere so we have shown $a | 2^m-1, 2^d-1 \iff a | 2^x-1$. Therefore $ 2^m-1, 2^d-1$ and $2^x-1$ have the same divisors and ultimately the same greatest one.
\end{document}

