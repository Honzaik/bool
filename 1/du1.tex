\documentclass[12pt, a4paper]{article}
\usepackage[margin=1in]{geometry}
\usepackage[utf8x]{inputenc}
\usepackage{indentfirst} %indentace prvního odstavce
\usepackage{mathtools}
\usepackage{amsfonts}
\usepackage{amsmath}
\usepackage{amssymb}
\usepackage{graphicx}
\usepackage{enumitem}
\usepackage{subfig}
\usepackage{float}
\usepackage[czech]{babel}
\usepackage{mathdots}
\usepackage{slashbox}

\begin{document}
\begin{center}
\large NMMB331 - HW1

\normalsize Jan Oupický
\end{center}
\vspace{1\baselineskip}

\section{}
We know that a boolean function $g: \mathbb{F}_{2^m} \rightarrow \mathbb{F}_2$ is balanced iff $\hat{g}(0) = 0$. Let $g(x) \coloneqq f(x)+f(x+u)$ for some $u \in \mathbb{F}_{2^{2n}}, u \neq 0$. We want to show, that $g(x)$ is balanced:
\begin{gather*}
\hat{g}(0) = \sum\limits_{x \in \mathbb{F}_{2^{2n}}} (-1)^{g(x)+\langle x,0 \rangle} = \sum\limits_{x \in \mathbb{F}_{2^{2n}}} (-1)^{g(x)} = \sum\limits_{x \in \mathbb{F}_{2^{2n}}} (-1)^{f(x)+f(x+u)}
\end{gather*}
Let's see what $f(x)$ and $f(x+u)$ is:
\begin{gather*}
f(x) = x_1 x_2 + x_3 x_4 + \dots x_{2n-1} x_{2n}\\
f(x+u) = (x_1+u_1) (x_2+u_2) + (x_3+u_3) (x_4+u_4) + \dots (x_{2n-1}+u_{2n-1}) (x_{2n}+u_{2n}) =\\
x_1 x_2 + x_1 u_2 + x_2 u_1 + u_1 u_2 + \text{same for others}
\end{gather*}
Therefore we can rewrite $f(x+u)$ as $f(x+u)=f(x)+f(u)+\langle x,u' \rangle$ where $u' = (u_2,u_1,u_3,u_4,\dots) \in \mathbb{F}_{2^{2n}}$ and since $u \neq 0$ then $u' \neq 0$. Therefore:
\begin{gather*}
\hat{g}(0) = \sum\limits_{x \in \mathbb{F}_{2^{2n}}} (-1)^{f(x)+f(x)+f(u)+\langle x,u' \rangle} = \sum\limits_{x \in \mathbb{F}_{2^{2n}}} (-1)^{f(u)+\langle x,u' \rangle} = (-1)^{f(u)}\sum\limits_{x \in \mathbb{F}_{2^{2n}}} (-1)^{\langle x,u' \rangle}
\end{gather*}
$(-1)^{f(u)}$ is $1$ or $-1$ but the last sum is $0$ by annihilator lemma since $u' \neq 0$. Thefore $f(x)+f(x+u)$ is balanced.

Or we can see as above that $g(x)$ is affine and use the rank nullity theorem which says that $\dim(Ker(g)) = 2^n-1$ and thefore half $x$ map to $0$ and the other half must map to $1$.
\section{}
a

\section{}
Let's denote $x = \gcd(m,d), y = \gcd(2^m-1, 2^d-1)$. We know that $y | 2^m-1$ and $y | 2^d-1 \implies 2^m \equiv 1 \, (y), 2^n \equiv 1 \, (y) \implies ord_{\mathbb{Z}_y}(2)|m, ord_{\mathbb{Z}_y}(2)|d \implies ord_{\mathbb{Z}_y}(2)|\gcd(m,d)=x$. Therefore $2^x \equiv 1 \, (y) \iff \gcd(2^m-1, 2^d-1) | 2^x-1$.

Let's denote $x = \gcd(m,d), y = \gcd(2^m-1, 2^d-1)$. Assume that $a | 2^m-1, 2^d-1 \iff 2^m \equiv 1 \, (a), 2^n \equiv 1 \, (a) \iff ord_{\mathbb{Z}_a}(2)|m,d \iff ord_{\mathbb{Z}_a}(2)|\gcd(m,d)=x \iff 2^x \equiv 1 \, (a)$. There have been equivalences everywhere so we have shown $a | 2^m-1, 2^d-1 \iff a | 2^x-1$. Therefore $ 2^m-1, 2^d-1$ and $2^x-1$ have the same divisors and ultimately the same greatest one.
\end{document}

