\documentclass[12pt, a4paper]{article}
\usepackage[margin=1in]{geometry}
\usepackage[utf8x]{inputenc}
\usepackage{indentfirst} %indentace prvního odstavce
\usepackage{mathtools}
\usepackage{amsfonts}
\usepackage{amsmath}
\usepackage{amssymb}
\usepackage{graphicx}
\usepackage{enumitem}
\usepackage{subfig}
\usepackage{float}
\usepackage[czech]{babel}
\usepackage{mathdots}
\usepackage{slashbox}
\newcommand{\qed}{\hfill\square}
\begin{document}
\begin{center}
\large NMMB331 - HW3

\normalsize Jan Oupický
\end{center}
\vspace{1\baselineskip}

\section{}
Let $x \in \mathbb{F}^n_2$. Then
\begin{gather*}
\frac{1}{2^n}\sum\limits_{u \in \mathbb{F}^n_2} F_g(u)(-1)^{\langle u, x \rangle} = \frac{1}{2^n}\sum\limits_{u \in \mathbb{F}^n_2} \sum\limits_{y \in \mathbb{F}^n_2}g(y)(-1)^{\langle u, y \rangle} (-1)^{\langle u, x \rangle} = \\
\frac{1}{2^n}\sum\limits_{y \in \mathbb{F}^n_2} g(y)\sum\limits_{u \in \mathbb{F}^n_2}(-1)^{\langle u, x+y \rangle}\\
\text{There is exactly one $y$ s.t. $x+y=0 \iff x=y$. Using annihilator lemma we get:}\\
\frac{1}{2^n}g(x)2^n + 0 = g(x)
\end{gather*}
$\qed$

\section{}
First assume $u=0$.
\begin{gather*}
F_f(0)=\sum\limits_{x \in \mathbb{F}^n_2}f(x)(-1)^{\langle 0,x \rangle} = \sum\limits_{x \in \mathbb{F}^n_2}f(x) = \sum\limits_{x \in \mathbb{F}^n_2, f(x)=1}1\\
2^{n-1}-\frac{1}{2}W_f(0) = \frac{1}{2}\left(\sum\limits_{x \in \mathbb{F}^n_2}(1-(-1)^{f(x)})\right) = \frac{1}{2}\left(\sum\limits_{x \in \mathbb{F}^n_2, f(x)=1}(1-(-1))\right)=\\
\frac{1}{2}\sum\limits_{x \in \mathbb{F}^n_2, f(x)=1}2 = F_f(0)
\end{gather*}
Now $u \neq 0$:
\begin{gather*}
F_f(u)=\sum\limits_{x \in \mathbb{F}^n_2}f(x)(-1)^{\langle u,x \rangle} = \sum\limits_{x \in \mathbb{F}^n_2,f(x)=0}0(-1)^{\langle u,x \rangle} + \sum\limits_{x \in \mathbb{F}^n_2,f(x)=1}1(-1)^{\langle u,x \rangle} =\\
\sum\limits_{x \in \mathbb{F}^n_2,f(x)=1}(-1)^{\langle u,x \rangle}\\
-\frac{1}{2}W_f(u) =\frac{1}{2}\left( \sum\limits_{x \in \mathbb{F}^n_2}(-1)^{\langle u,x \rangle} - W_f(u) \right) = \frac{1}{2}\left( \sum\limits_{x \in \mathbb{F}^n_2}(-1)^{\langle u,x \rangle} - (-1)^{f(x)+\langle u,x \rangle} \right) = \\
\frac{1}{2}\left( \sum\limits_{x \in \mathbb{F}^n_2}(-1)^{\langle u,x \rangle}(1 - (-1)^{f(x)} \right) = \frac{1}{2}\left( \sum\limits_{x \in \mathbb{F}^n_2, f(x)=0}(-1)^{\langle u,x \rangle}(1 - 1) \right) + \\
\frac{1}{2}\left( \sum\limits_{x \in \mathbb{F}^n_2, f(x)=1}(-1)^{\langle u,x \rangle}(1 - (-1)) \right) = 0 + F_f(u) = F_f(u)
\end{gather*}
$\qed$

\section{}
\begin{gather*}
F_{f \oplus g}(u) = \sum\limits_{x \in \mathbb{F}^n_2}(f \oplus g)(x)(-1)^{\langle u,x \rangle} = \sum\limits_{x \in \mathbb{F}^n_2}(f(x) \oplus g(x))(-1)^{\langle u,x \rangle} = \\
\sum\limits_{x \in \mathbb{F}^n_2}(f(x) + g(x) - 2f(x)g(x))(-1)^{\langle u,x \rangle} = \\
\sum\limits_{x \in \mathbb{F}^n_2}f(x)(-1)^{\langle u,x \rangle} + \sum\limits_{x \in \mathbb{F}^n_2}g(x)(-1)^{\langle u,x \rangle} + \sum\limits_{x \in \mathbb{F}^n_2}2f(x)g(x)(-1)^{\langle u,x \rangle} = \\
F_f(u) + F_g(u) + \sum\limits_{x \in \mathbb{F}^n_2}2(fg)(x)(-1)^{\langle u,x \rangle} = F_f(u) + F_g(u) + F_{2fg}(u)
\end{gather*}
$\qed$

\section{}
\end{document}

