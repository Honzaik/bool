\documentclass[12pt, a4paper]{article}
\usepackage[margin=1in]{geometry}
\usepackage[utf8x]{inputenc}
\usepackage{indentfirst} %indentace prvního odstavce
\usepackage{mathtools}
\usepackage{amsfonts}
\usepackage{amsmath}
\usepackage{amssymb}
\usepackage{graphicx}
\usepackage{enumitem}
\usepackage{subfig}
\usepackage{float}
\usepackage[czech]{babel}
\usepackage{mathdots}
\usepackage{slashbox}
\newcommand{\qed}{\hfill\square}
\begin{document}
\begin{center}
\large NMMB331 - HW4

\normalsize Jan Oupický
\end{center}
\vspace{1\baselineskip}

\section{}
\begin{enumerate}
    \item $\sim_{EA}$ is reflexive since we can take $A=id=B$ (identities) and $C = 0$ (null map).
    \item If $A,B$ are affine permutations and $C$ is an affine map s.t. $G(x)= A \circ F \circ B (x) + C(x) \iff F \sim_{EA} G$ then we can write $A^{-1}\circ G \circ B^{-1} (x) + A^{-1}\circ C \circ B^{-1} (x) = F(x) \iff G \sim_{EA} F$. Since $A,B$ are affine permutations then their inverses are also affine permutations and also $A^{-1}\circ C \circ B^{-1}$ is affine since $C$ is affine and $A^{-1},B^{-1}$ are affine permutations. This proves symmetry.
    \item Assume $F \sim_{EA} G$, $G \sim_{EA} H$ and we want to show $F \sim_{EA} H$. By definition we have $G(x)= A_1 \circ F \circ B_1 (x) + C_1(x)$ and $H(x) = A_2 \circ G \circ B_2 (x) + C_2(x)$. $A_2(x) = M(x)+y$ for some $M$ linear permutation and $y$ vector.
    \begin{gather*}
    H(x) = A_2 \circ G \circ B_2 (x) + C_2(x) = A_2 (A_1 \circ F \circ B_1 (B_2(x)) + C_1(B_2(x))) + C_2(x) =\\
    M(A_1 \circ F \circ B_1 (B_2(x)) + C_1(B_2(x))) + y + C_2(x) =\\
    M(A_1 \circ F \circ B_1 (B_2(x))) + M(C_1(B_2(x))) + y + C_2(x) = \\
    (M \circ A_1) \circ F \circ (B_1 \circ B_2)(x) + (M \circ C_1 \circ B_2(x) + y + C_2(x))
    \end{gather*}
    $M \circ A_1$ is an affine permutation since $M$ is linear permutation and $A_1$ is affine permutation. $B_1 \circ B_2$ is affine permutation since both are affine permutations. $M \circ C_1 \circ B_2$ is affine map since $M, B_2$ are affine permutations and $C_1$ affine map, $C_2(x)+y$ is an affine map. Sum of affine maps is affine map. This proves transitivity.
\end{enumerate}
$\qed$

\section{}
A point $(x,F(x))$ on $\Gamma_F$ corresponds with a point $(F(x),x)$ on $\Gamma_{F^{-1}}$ since $F^{-1}(F(x)) = x$. Therefore our affine permutation should just switch those two "coordinates". 

Using the notation mentioned we can set $A,D = 0$ (zero matrix) and $B,C$ to be identity matrices of dimension $n$. The affine parts $u,v$ are also zero vectors. This map is clearly an affine permutation and as said before it maps $\Gamma_{F^{-1}} = \mathcal{A}(\Gamma_F)$.
$\qed$

\section{}
Since $G \sim_{CCZ} F$ we know that for a $x \in \mathbb{F}_2^n$ there exists $y \in \mathbb{F}_2^n$ s.t.:
\begin{gather*}
\begin{pmatrix}
y\\
G(y)
\end{pmatrix}
= \begin{pmatrix}
A & 0\\
C & D
\end{pmatrix}
\begin{pmatrix}
x\\
F(x)
\end{pmatrix}
+ \begin{pmatrix}
u\\
v
\end{pmatrix}
\\
\iff\\
y = Ax + u\\
G(y) = Cx + (D \circ F)(x) + v
\end{gather*}
Since $\mathcal{A}$ is an affine permutation it must be that the map $M: x \mapsto Ax+u$ is invertible i.e. $x = A^{-1}(y+u) = A^{-1}(y)+A^{-1}(u)$ where $A^{-1}$ is the matrix inverse of $A$. Therefore the inverse map $M^{-1}: y \mapsto A^{-1}y + A^{-1}u$ is also an affine permutation. Now we will just apply it to our equality above.
\begin{gather*}
G(y) = Cx + (D \circ F)(x) + v \iff (G \circ M)(x) = Cx + (D \circ F)(x) + v\\
\text{Apply inverse $M^{-1}$:}\\
G(x) = (C \circ M^{-1})(x) + (D \circ F \circ M^{-1})(x) + M^{-1}v\\
G(x) = (D \circ F \circ M^{-1})(x) + ((C \circ M^{-1})(x) + M^{-1}v)
\end{gather*}
We know that $M^{-1}$ is an affine permutation . $C$ is a linear map and therefore the composition of $M^{-1}$ and $C$ is also an linear map and together with the constant vector $M^{-1}v$ they form an affine map. 

Now we just need to show that $D$ is an affine permutation to fullfil the $EA$ requirements. Since $\mathcal{A} = \mathcal{L} + (u,v)$ and $\mathcal{A}$ is an affine permutation and $\mathcal{L}$ is linear, it must be that $\mathcal{L}$ is a linear permuation. Since $\mathcal{L}$ can be represented as a matrix and is a permutation that means that every subset of it columns must be linearly independent. Choose the subset of columns containing $D$. They must be linearly independent and this implies that columns of $D$ must be linearly independent since $B = 0$. This means that $D$ is a linear permutation.
$\qed$

\section{}
Since $F \sim_{CCZ}G$ where $u,v=0$ we can write that for some $x \in \mathbb{F}_2^n: y = F_1(x)$ and $G(y)=F_2(x)$. We want to show that there exists 1 to 1 correpondence between $(a,b) \in \mathbb{F}_2^n \times \mathbb{F}_2^n $ and some $(a',b') \in \mathbb{F}_2^n \times \mathbb{F}_2^n$ s.t. $\hat{G}(a,b) = \hat{F}(a',b')$.
\begin{gather*}
\hat{G}(a,b) = \sum\limits_{y \in \mathbb{F}_2^n} (-1)^{\langle a,G(y) \rangle + \langle b, y \rangle} =  \sum\limits_{x \in \mathbb{F}_2^n} (-1)^{\langle a,F_2(x) \rangle + \langle b, F_1(x) \rangle}\\
\text{Lets focus on the exponent now:}\\
\langle a,F_2(x) \rangle + \langle b, F_1(x) \rangle = \langle a,Cx + D(F(x)) \rangle + \langle b, Ax + B(F(x)) \rangle = \\
\langle a,Cx \rangle + \langle a,D(F(x)) \rangle + \langle b, Ax \rangle  + \langle b,B(F(x)) \rangle =\\
\text{$A,B,C,D$ are all linear so we will use their adjoint maps:}\\
= \langle C^*a,x \rangle + \langle D^*a,F(x) \rangle + \langle A^*b, x \rangle  + \langle B^*b,F(x) \rangle = \\
\langle C^*a + A^*b, x \rangle + \langle D^*a + B^*b, F(x) \rangle \implies\\
\hat{G}(a,b) = \hat{F}(a',b') \text{ where } a' = C^*a + A^*b, b' = D^*a + B^*b\\
\text{In matrix notation:}\\
\begin{pmatrix}
a'\\
b'
\end{pmatrix} =
\begin{pmatrix}
A^* & C^*\\
B^* & D^*
\end{pmatrix}\begin{pmatrix}
b\\
a
\end{pmatrix} \iff (a',b') = \mathcal{L}^*(b,a)
\end{gather*}
Since we know that $\mathcal{L}^*$ is a permutation then we have the 1 to 1 correspondence between $(a,b)$ and $(a',b')$.
$\qed$
\end{document}

